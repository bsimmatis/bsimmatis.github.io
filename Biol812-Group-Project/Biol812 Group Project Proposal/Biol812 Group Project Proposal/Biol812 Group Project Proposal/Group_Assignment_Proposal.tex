\documentclass[]{article}
\usepackage{lmodern}
\usepackage{amssymb,amsmath}
\usepackage{ifxetex,ifluatex}
\usepackage{fixltx2e} % provides \textsubscript
\ifnum 0\ifxetex 1\fi\ifluatex 1\fi=0 % if pdftex
  \usepackage[T1]{fontenc}
  \usepackage[utf8]{inputenc}
\else % if luatex or xelatex
  \ifxetex
    \usepackage{mathspec}
  \else
    \usepackage{fontspec}
  \fi
  \defaultfontfeatures{Ligatures=TeX,Scale=MatchLowercase}
\fi
% use upquote if available, for straight quotes in verbatim environments
\IfFileExists{upquote.sty}{\usepackage{upquote}}{}
% use microtype if available
\IfFileExists{microtype.sty}{%
\usepackage{microtype}
\UseMicrotypeSet[protrusion]{basicmath} % disable protrusion for tt fonts
}{}
\usepackage[margin=1in]{geometry}
\usepackage{hyperref}
\hypersetup{unicode=true,
            pdftitle={Biol812: Project Proposal},
            pdfauthor={C. Juurakko, N. Kharazian (auditing), S. Knebel, B. Simmatis, L. Simmatis (auditing)},
            pdfborder={0 0 0},
            breaklinks=true}
\urlstyle{same}  % don't use monospace font for urls
\usepackage{graphicx,grffile}
\makeatletter
\def\maxwidth{\ifdim\Gin@nat@width>\linewidth\linewidth\else\Gin@nat@width\fi}
\def\maxheight{\ifdim\Gin@nat@height>\textheight\textheight\else\Gin@nat@height\fi}
\makeatother
% Scale images if necessary, so that they will not overflow the page
% margins by default, and it is still possible to overwrite the defaults
% using explicit options in \includegraphics[width, height, ...]{}
\setkeys{Gin}{width=\maxwidth,height=\maxheight,keepaspectratio}
\IfFileExists{parskip.sty}{%
\usepackage{parskip}
}{% else
\setlength{\parindent}{0pt}
\setlength{\parskip}{6pt plus 2pt minus 1pt}
}
\setlength{\emergencystretch}{3em}  % prevent overfull lines
\providecommand{\tightlist}{%
  \setlength{\itemsep}{0pt}\setlength{\parskip}{0pt}}
\setcounter{secnumdepth}{5}
% Redefines (sub)paragraphs to behave more like sections
\ifx\paragraph\undefined\else
\let\oldparagraph\paragraph
\renewcommand{\paragraph}[1]{\oldparagraph{#1}\mbox{}}
\fi
\ifx\subparagraph\undefined\else
\let\oldsubparagraph\subparagraph
\renewcommand{\subparagraph}[1]{\oldsubparagraph{#1}\mbox{}}
\fi

%%% Use protect on footnotes to avoid problems with footnotes in titles
\let\rmarkdownfootnote\footnote%
\def\footnote{\protect\rmarkdownfootnote}

%%% Change title format to be more compact
\usepackage{titling}

% Create subtitle command for use in maketitle
\newcommand{\subtitle}[1]{
  \posttitle{
    \begin{center}\large#1\end{center}
    }
}

\setlength{\droptitle}{-2em}
  \title{Biol812: Project Proposal}
  \pretitle{\vspace{\droptitle}\centering\huge}
  \posttitle{\par}
\subtitle{Broadening the spectra: looking beyond VRS-inferred chlorophyll a
(650-700nm) to identify changes in whole-lake primary production.}
  \author{C. Juurakko, N. Kharazian (auditing), S. Knebel, B. Simmatis, L.
Simmatis (auditing)}
  \preauthor{\centering\large\emph}
  \postauthor{\par}
  \predate{\centering\large\emph}
  \postdate{\par}
  \date{March 18, 2018}


\begin{document}
\maketitle

\section{Team name: The Biomath 5}\label{team-name-the-biomath-5}

\section{Introduction}\label{introduction}

Water quality information is critical in assessing lake health, but
little long-term monitoring data are available to determine whether
modern conditions are within a range of pre-impact (e.g.~prior to
eutrophication) variability. Typically, impacted lake systems are
monitored after the impact has occurred, resulting in possibly
optimistic, but arguably unrealistic, management strategies. In Canada,
regular lake monitoring typically only covers the past few decades (if
present at all) and is often initiated in response to detection of an
environmental problem, and as a result aims to answer how have humans
altered conditions in a system after a specific event (e.g.~a
cyanobacterial bloom). Given the paucity of monitoring data,
understanding how, where and when anthropogenic activities have changed
aquatic environments beyond their natural range of variability is
difficult, as many impacts are regionally or locally specific Michelutti
et al. (2009).

\section{Data Collection and
Description}\label{data-collection-and-description}

Paleolimnological approaches use a variety of biological, physical and
chemical indicators preserved in lake sediments to reconstruct past
environmental conditions (Smol, 2008). Visible range spectroscopy (VRS)
can track past changes in whole lake primary production by tracking
sedimentary chlorophyll a and its degradation products, providing
information on shifts in lake trophic status (Michelluti et al., 2010).
Most cores were collected and had the sediment-water interface
stabilized with Zorbitol© before refrigerated shipment to Université
Laval, Laval, QC. Cores were then split and stored in a cold room
(\textless{}4°C) prior to shipment to PEARL, Queen's University,
Kingston, ON. Individual intervals of freeze-dried sediment have been
dated via gamma spectroscopy at Queen's University using unsupported
210Pb activities in a constant rate of supply (CRS) model (Appleby and
Oldfield, 1978; Binford 1990). Freeze-dried sediments have been sieved
(120µm mesh) and sub-sampled such that a 50mL glass scintillation vial
has at least 1mm of sediment. The sediments have been scanned through
the base of the glass vials using a Rapid Content Analyzer (FOSS
NIRSystems Inc.) operating over a range of 400-2,500 nm.

The data is comprised of a series of absorbance values from 400-2,500nm
through time, which broadly represents the proportion of organic matter
in the sediment. The absorbance values demonstrate the type of algae
present in the lake depending on where absorbance peaks through the
spectra (e.g.~chlorophyll a is expected to occur between 650-700nm,
representing whole-lake primary production). Each sediment interval has
2100 values and each lake core can have over 30 intervals of sediment
stored in 2D datasets. Our largest data set, Muskrat Lake, has 71
intervals of sediment with 2100 values per interval for a total of
\textasciitilde{}72 000 data points.

The simulated data will be generated according to a variation on
Agent-based modelling (ABM) that is commonly used in ecology, referred
to as invidivual-based modelling (IBM) . In particular, we shall use the
parameter of absorbance to simulate the evolution of an algae population
in a lake according to individual behavior. The data obtained will be
used to make predictions about how the enviroment will change according
to varying levels of absorbance.

\section{Sites}\label{sites}

Stoco Lake was classified as eutrophic (spring TP ranged from \(30-50\)
\(\mu g\cdot L^{-1}\)) until sewage management upgrades were installed
in the late 1980's. Currently, the lake is mesotrophic (spring TP
\(\sim 15\) \(\mu g\cdot L^{-1}\), maximum depth \(12m\)), though
cyanobacterial blooms and excessive summer algal growth still occur
frequently, likely due to the internal loading of phosphorus from the
sediments.

Lac Duhamel (maximum depth \(26m\); oligotrophic, summer TP \(<4\)
\(\mu g\cdot L^{-1}\)) is located upstream of Mont Tremblant, Quebec,
has records of high human impact, including forestry and highway road
salt application. It is extremely sensitive to cultural eutrophication,
primarily due to the channelization of its tributaries, Shield bedrock,
and long water residence time of 2 years (Biofilia, 2004).

Lac-des-Îles (maximum depth \(37m\)) is located on the Canadian Shield
near Mont-Laurier, Quebec, and is classified as oligo-mesotrophic
(summer TP \(\sim 10.5\) \(\mu g\cdot L^{-1}\); Enviro'Eau, 2009). The
lake supports a variety of sports fish, including lake trout, and is
generally considered environmentally sensitive to eutrophication due to
its relatively low nutrient levels.

Muskrat Lake (maximum depth \(60m\)) is valued for its recreational use
and as a drinking water source for the town of Cobden, Ontario, located
at the south end of the lake. High total phosphorus (\(>40\)
\(\mu g/L\)), low water clarity, and algal blooms have raised water
quality concerns, largely attributed to eutrophication from heavy
agricultural watershed development.

\section{Research Questions}\label{research-questions}

We aim to determine whether (1) chlorophyll a changes over time in each
lake, and (2) we can ``fingerprint'' deep vs shallow eutrophic lakes,
and, if so, whether the differences correlate to the wavelengths of
known specific pigments.

\section{Significance}\label{significance}

This study aims to explore the environmental histories of previously
unmonitored lakes to inform local lake management decisions regarding
water quality. Further, absorbance has been previously used to
quantitatively reconstruct chlorophyll a, but other spectra have not
been fully explored despite the potential use for reconstructing
soft-bodied algal communities (i.e.~cyanobacteria using genus-specific
pigments such as echinenone). This information can be further used to
determine if a lake system has reached an ecological tipping point and
fundamentally changed from pre-eutrophication conditions.

\section{Pipeline}\label{pipeline}

Our analysis will plot and determine differences in absorbance values
through time. Each member of our team will be responsible for a
different module (Figure 1). Bash script will be used to access and
organize all files, as well as generate README files and text
descriptions of the data sets. R and/or Python will be used for the
remainder of the processing. GitHub will be used to pull raw data, store
text-based files (for example, the text for the poster), and for general
communication or troubleshooting of code. See appendix for visual
representation.

Brigitte will be responsible for the module relating to reconstructing
chlorophyll a, plotting and processing the raw absorbance data, and
determining the VRS regions with the greatest variance through time.
This will involve the creation of functions for use in loops to generate
the inferred chlorophyll a data, creation of a function or other program
to determine the regions with the most variance and plotting of data.

Collin will be responsible for accessing climate data using Python or R
and plotting the georeferenced coring locations using spatial ecology
approaches. He is also responsible for plotting climate data,
determining its change over time and accessing it appropriately using
Python or R.

Stefanie will be responsible for creating a mathematical model and
programming a simulation in python based on that model. She will run the
simulation mutliple times while varying the parameters of the model and
plot one key finding. She will also work with R-markdown and LaTeX to
create a poster for everyone's work to be included.

Leif will be creating and applying machine learning techniques to
identify and contrast VRS spectra for each lake, with the goal of
investigating whether the depth or trophic status can be identified from
only the absorbance values.

Nazila will observe each component of the project to ensure analysis and
visualizations are informative and void of extraneous `ink'. She will
also assist Stefanie with her python code and will make comparisons with
R script.

\section{Predicted or hypothesized
results}\label{predicted-or-hypothesized-results}

\section{References}\label{references}

Appleby, P.G., and Oldfield, F. 1978. The calculation of lead-210 dates
assuming a constant rate of supply of unsupported 210Pb to the sediment.
Catena, 5:1-8.

Binford, M. 1990. Calculation and uncertainty analysis of 210Pb dates
for PIRLA project lake sediment cores. Journal of Paleolimnology, 3(3):
253-267.

Biofilia (Consultants en Environnement). 2004. Diagnose de basin
versant: Lac Duhamel, Ville de Mont-Tremblant. Rapport final (only
available in French), URL:
\url{http://lacduhamel.ca/wp-content/uploads/2015/09/diagnose-bassin-versant.pdf},
accessed 15 Nov 2017.

Enviro'Eau (Services-Conseils). 2009. Diagnose du Lac- des- Îles:
Municipalité de Sain-Aimé-du-Lac-des- Îles et Ville de Mont-Laurier,
Québec. Rapport final (only available in French), URL:
www.villemontlaurier.qc.ca/DATA/DOCUMENT/Rapport et annexes-diagnose lac
des Îles-2009.pdf, accessed 9 Nov 2017.

\begin{figure}
\centering
\includegraphics{Figure 1.eps}
\caption{General workflow and areas of responsibility for data
processing.}
\end{figure}

\begin{figure}
\centering
\includegraphics{Figure 2.eps}
\caption{General workflow and areas of responsibility for data
processing.}
\end{figure}

\hypertarget{refs}{}
\hypertarget{ref-Michelutti_2009}{}
Michelutti, Neal, Jules M. Blais, Brian F. Cumming, Andrew M. Paterson,
Kathleen Rühland, Alexander P. Wolfe, and John P. Smol. 2009. ``Do
Spectrally Inferred Determinations of Chlorophyll a Reflect Trends in
Lake Trophic Status?'' \emph{Journal of Paleolimnology} 43 (2). Springer
Nature: 205--17.
doi:\href{https://doi.org/10.1007/s10933-009-9325-8}{10.1007/s10933-009-9325-8}.


\end{document}
